\documentclass[german]{lni}

%\IfFileExists{latin1.sty}{\usepackage{latin1}}{\usepackage{isolatin1}}
\usepackage[utf8]{inputenc}
\usepackage[ngerman]{babel}
\usepackage{graphicx}
\usepackage{url}
\author{B.Sc. Dominic Dahnelt, dominic.dahnelt@stud.hs-flensburg.de\\
B.Sc. Peter Steensen, peter.steensen@stud.hs-flensburg.de}
\title{Cross Platform Development mit Ionic}


\begin{document}
\maketitle

\begin{abstract}
	Ionic ist ein Framework, welches mehrere Technologien miteinander verbindet um Cross-Platform-Development zu betreiben. Dazu wird eine App als Webanwendung entwickelt und in einem abschließenden Schritt zu einer nativen App kompiliert.
\end{abstract}

\section{Literatur}
\subsection{Theorie/Verfahren}
Ionic2 ist das Bindeglied zwischen HTML5, AngularJS 2 und Cordova und ermöglicht dadurch die Entwicklung einer Webapplikation, die im späteren Verlauf als native App für verschiedene Betriebssyteme deployed werden kann. Durch die Verwendung von Javascript können auch sonstige Javascript-Frameworks eingebunden werden.
\\
Im Vergleich zu nativer App Entwicklung für Android  werden die Views als HTML-Templates erstellt und dynamisch mit Inhalten gefüllt. Der Inhalt wird dabei durhc Javascript-Code generiert.
\par 
\url{https://ionicframework.com/docs/}

\subsection{Tutorials}
TODO
\subsection{Projekt-/Sourcequellen}
TODO

\section{Dokumentation der Beispielprojekte}
\subsection{Portierung des ToDo-Managers}
Folgende Schritte wurden durchgeführt, um den ToDo-Manager für ionic umzuschreiben:
\begin{enumerate}
	\item Ionic-App mit dem Template \emph{tabs} erstellen
	\item App in app.components.ts und app.modules.ts entsprechend benennen (anstelle von MyApp)
	\item Die Standard-Seiten \emph{home}, \emph{about} und \emph{contact} entfernen und die Verknüpfung aus der tabs.ts entfernen
	\item Die Seiten \emph{task} und \emph{list} hinzufügen (.ts und .html) und in der tabs.ts verknüpfen
	\item Das HTML-Template für task hinzufügen
	\item Ein Menü in der app.html hinzufügen und einen entsprechenden Toggle-Button implementieren
	\item Flag \emph{persistent=true} setzen, damit das Menu nicht nur auf der Root-Page sichtbar ist
	\item In der app.html die Überschrift \emph{TODOS} hinzufügen
	\item ToastController in task.ts injecten
	\item Toast-Ausgabe bei Button-Click implementieren
	\item Error-Handling implementieren, falls kein Titel für ein ToDo angegeben wurde
	\item Plugininstallation von cordova-sqlite-storage	
	\item Ng-Module hinzufügen
	\item Dependency in task injecten
\end{enumerate}

\subsection{Laboraufgabe}
TODO
\section{Kritische Bewertung}
TODO

%\bibliography{literature}
\end{document}





